\documentclass{report}

\title{
    Relazione di progetto di Tecnologie Web\\
    \large Sito web di e-commerce per la vendita di prodotti per l'agricoltura
}

\author{Federico Bagattoni, Luca Venturini, Jacopo Turchi}

\date{\today}

\begin{document}

\maketitle

\tableofcontents

\chapter{Design}

\section{Creazione e discussione dei mock up}
\p{
    Viene deciso di creare un mock-up per ogni membro del gruppo che poi verrà discusso, durante una sessione di \textit{experience prototyping}. Successivamente verranno valutate le modifiche suggerite dal gruppo di utenti e con queste verrà prodotto un nuovo mock-up.
} 
\p{
    Dopo aver realizzato e confrontato i mock-up si giunge alla conclusione che: 
}
\begin{itemize}
    \item {Per quanto riguarda la pagina principale vengono gradite sia l'idea in mock-up 1 di una foto che occupi tutto lo spazio orizzontale sia l'idea in mock-up 2 di avere un \textit{carosello} di immagini nel gergo detto \textit{Jumbotron}. Quindi si propone un compromesso tra i due}
    \item {Viene apprezzata la modularità espressa dal mock-up 2 che consente ampia scalabilità senza aumentare la complessità di lettura del sito. Al contrario di mock-up 1 che, tenendo un immagine intera come sfondo, limita il numero di informazioni che possono essere mostrate.}
    \item {Alcuni utenti trovano difficoltà nel distinguere le icone dallo sfondo nel mock-up 1 perché il disegno di sfondo rende complicata la lettura, preferendo la soluzione del mock-up 2.}
    \item {Il footer del mock-up 1 è stato scelto come il più adeguato in quanto è coerente con il contesto del sito. In particolare osserviamo che la presenza del \textbf{grano} nello sfondo viene apprezzata dagli utenti.}
    \item {E' stato motivo di discussione la posizione dell'\textit{hamburger menù} con la netta maggioranza degli utenti che lo trova più comodo in alto a sinistra anche se l'idea iniziale era posizionarlo in alto a destra visto che la popolazione è prevalentemente destrorsa.}
    \item {Del mock-up 3 è stato apprezzato il modo di gestire le categorie, disponendole verticalmente una sotto l'altra.}
    \item {E' stata inoltre valutata l'idea della navbar del mock-up 3, però agli occhi degli utenti era sembrata un'organizzazione più disordinata rispetto a quella del mock-up 2.}
    //TO-DO: pagina prodotto
\end{itemize}
\p {
    Dopo aver discusso i punti precedenti viene creato un nuovo mock-up contenente le modifiche suggerite, che viene gradito da tutti i membri del gruppo.
}

\end{document}